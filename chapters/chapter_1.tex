\chapter{
    مقدمه‌ای بر هوش مصنوعی
}

در این فصل، سعی میکنیم یه این سوال پاسخ دهیم: «هوش مصنوعی چیست و چرا باید آن را مطالعه کنیم؟»

\section{تعریف هوش مصنوعی (چهار رویکرد)}
\begin{itemize}
    \item[الف)] رویکردهای انسان‌محور\LTRfootnote{Human-centered}
    

    هدف این است که ماشین‌ها را شبیه به «ما» بسازیم.
    \begin{enumerate}
        \item \textbf{انسانی رفتار کردن }(آزمون تورینگ): آلن تورینگ در سال ۱۹۵۰
        پیشنهادی داد: اگر یک داور انسانی نتواند در یک گفتگوی متنی تشخیص دهد که
        طرف مقابلش انسان است یا ماشین، آن ماشین هوشمند است.
        
        \begin{itemize}
            \item[*] نیازها: پردازش زبان طبیعی، بازنمایی دانش، استدلال
            خودکار و یادگیری ماشینی.
        \end{itemize}

        \item \textbf{انسانی فکر کردن }(مدل‌سازی شناختی): در اینجا ما می‌خواهیم بدانیم ذهن انسان چگونه کار می‌کند (مثلاً از طریق تصویربرداری مغزی یا آزمایش‌های روانشناسی) و همان مدل را در کامپیوتر پیاده کنیم.
    \end{enumerate}

    \item[ب)] رویکردهای عقلانی\LTRfootnote{Rationalist}
    

    هدف این است که ماشین «کار درست» را انجام دهد، حتی اگر شبیه انسان نباشد.
    \begin{enumerate}
        \item \textbf{عقلانی فکر کردن }(قوانین منطق): استفاده از ارسطو و منطق صوری. یعنی اگر بگوییم «همه انسان‌ها فانی هستند» و «سقراط انسان است»، ماشین باید نتیجه بگیرد «سقراط فانی است».
        \begin{itemize}
            \item[*] مشکل: بیان دانش غیرقطعی با منطق خشک ریاضی بسیار دشوار است.
        \end{itemize}
        \item \textbf{عقلانی رفتار کردن }(عامل عقلانی\LTRfootnote{Rational Agent}): این رویکرد اصلی ما در این کتاب است.
        \begin{itemize}
            \item[*]{
                تعریف: یک عامل عقلانی طوری عمل می‌کند که به «بهترین نتیجه» (یا بهترین نتیجه‌ی احتمالی) برسد.
            }
            \item[*]{
                چرا این رویکرد بهتر است؟ چون کلی‌تر از منطق است (گاهی لازم است بدون استدلال منطقی و صرفاً بر اساس غریزه یا احتمال عمل کرد) و از نظر علمی هم تعریف دقیق‌تری نسبت به تقلید از رفتار پیچیده و گاهی غیرعقلانی انسان دارد.
            } 
        \end{itemize}
    \end{enumerate}
\end{itemize}

\section{
    مبانی هوش مصنوعی (ریشه‌های تاریخی)
}
هوش مصنوعی یک‌باره از آسمان نیفتاده است! این علم وام‌دار حوزه‌های زیر است:
\begin{itemize}
    \item[*]{
        \textbf{فلسفه}: فیلسوفانی مثل ارسطو و دکارت پرسیدند: آیا ذهن از قوانین فیزیکی پیروی می‌کند؟ دانش از کجا می‌آید؟
    }
    \item[*]{
        \textbf{ریاضیات}: منطق (بول)، احتمالات (بیز) و الگوریتم‌ها (خوارزمی و تورینگ) ابزارهای اصلی AI هستند.
    }
    \item[*]{
        \textbf{اقتصاد}: مفاهیم «سودمندی»\LTRfootnote{Utility} و «نظریه بازی‌ها» به ما یاد دادند چگونه در شرایطی که دیگران هم تصمیم‌گیرنده هستند، بهترین انتخاب را بکنیم.
    }
    \item[*]{
        \textbf{علوم اعصاب}: مطالعه اینکه مغز چگونه اطلاعات را پردازش می‌کند (الهام‌بخش شبکه‌های عصبی).
    }
    \item[*]{
        \textbf{روانشناسی}: مطالعه ادراک و یادگیری انسان.
    }
\end{itemize}
\section{
    تاریخچه هوش مصنوعی (خلاصه ادوار)
}
دانستن تاریخ به ما کمک می‌کند اشتباهات گذشته را تکرار نکنیم:
\begin{enumerate}
    \item {
        تولد (۱۹۴۳-۱۹۵۶): اولین مدل‌های عصبی و نشست مشهور «دارتموث» که نام AI در آنجا انتخاب شد.
    }
    \item {
        اشتیاق اولیه و پیش‌بینی‌های خوش‌بینانه (۱۹۵۲-۱۹۶۹): زمانی که فکر می‌کردند تا ۱۰ سال دیگر هوش مصنوعی از انسان جلو می‌زند.
    }
    \item {
        دوزخ واقعیت (اولین زمستان \lr{AI}): وقتی محققان فهمیدند حل مسائل دنیای واقعی با منطق ساده ممکن نیست و بودجه‌ها قطع شد.
    }
    \item {
        سیستم‌های خبره (دهه ۸۰): هوش مصنوعی وارد صنعت شد (برنامه‌هایی که دانش پزشکان یا مهندسان را کپی می‌کردند).
    }
    \item {
        بازگشت شبکه‌های عصبی و یادگیری عمیق (۲۰۰۰ تا کنون): با آمدن داده‌های عظیم\LTRfootnote{Big Data} و سخت‌افزارهای قدرتمند (GPU)، هوش مصنوعی وارد دوران طلایی فعلی شد.
    }
\end{enumerate}
\section[وضعیت فعلی]{
    وضعیت فعلی\LTRfootnote{State of the Art}
}
از توانایی های خیره کننده AI می توان به مواد زیر اشاره کرد:
\begin{itemize}
    \item[*]{
        شکست دادن قهرمانان جهان در بازی‌های پیچیده (مثل \lr{Go}).
    }
    \item[*]{
        ترجمه هم‌زمان زبان‌ها.
    }
    \item[*]{
        ماشین‌های خودران.
    }
    \item[*]{
        دستیارهای صوتی و مدل‌های زبانی بزرگ (مثل چت با \lr{chatGPT} و \lr{gemini}).
    }
\end{itemize}

\section*{
    نکته کلیدی برای درک ادامه کتاب
}
کل کتاب بر پایه مفهوم «عامل»\LTRfootnote{Agent} بنا شده است. عامل چیزی است که:
\begin{enumerate}
    \item محیط را حس می‌کند.\LTRfootnote{Perceive}
    \item روی محیط اثر می‌گذارد.\LTRfootnote{Act}
    \item هدفش بیشینه‌سازی عقلانیت است.
\end{enumerate}

\section*{سخن پایانی}
هوش مصنوعی فقط ساختن ربات‌های شبیه انسان نیست؛ بلکه علم طراحی «عامل‌های هوشمندی» است که بتوانند در محیط‌های پیچیده، بهترین تصمیمات را برای رسیدن به هدف بگیرند.