\chapter{
    مقدمه‌ای بر هوش مصنوعی
}

در این فصل، سعی میکنیم یه این سوال پاسخ دهیم: «هوش مصنوعی چیست و چرا باید آن را مطالعه کنیم؟»

\section{تعریف هوش مصنوعی (چهار رویکرد)}
\begin{itemize}
    \item[الف)] رویکردهای انسان‌محور (Human-centered):
    هدف این است که ماشین‌ها را شبیه به «ما» بسازیم.
    \begin{enumerate}
        \item \textbf{انسانی رفتار کردن }(آزمون تورینگ): آلن تورینگ در سال ۱۹۵۰
        پیشنهادی داد: اگر یک داور انسانی نتواند در یک گفتگوی متنی تشخیص دهد که
        طرف مقابلش انسان است یا ماشین، آن ماشین هوشمند است.
        
        \begin{itemize}
            \item[*] نیازها: پردازش زبان طبیعی، بازنمایی دانش، استدلال
            خودکار و یادگیری ماشینی.
        \end{itemize}

        \item \textbf{انسانی فکر کردن }(مدل‌سازی شناختی): در اینجا ما می‌خواهیم بدانیم ذهن انسان چگونه کار می‌کند (مثلاً از طریق تصویربرداری مغزی یا آزمایش‌های روانشناسی) و همان مدل را در کامپیوتر پیاده کنیم.
    \end{enumerate}

    \item[ب)] 
\end{itemize}